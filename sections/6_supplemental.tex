\section{Supplemental Material}
\label{sec:supplemental}

\subsection{Additional Confidence Histograms}
We provide extended histograms for all evaluated models, comparing the [0-1] and [1-10] prompting scales.

\begin{figure*}[h]
    \centering
    \includegraphics[width=0.32\textwidth]{figures/self_confidence_score_scale/o1_histogram_uncalibrated_Gemini 2.5 Flash.png}
    \includegraphics[width=0.32\textwidth]{figures/self_confidence_score_scale/o1_histogram_uncalibrated_Gemini 2.0 Flash.png}
    \includegraphics[width=0.32\textwidth]{figures/self_confidence_score_scale/o1_histogram_uncalibrated_GPT-5 Mini Low.png}
    \caption{Confidence score distributions for Gemini 2.5 Flash, Gemini 2.0 Flash, and GPT-5 Mini Low. The [0-1] scale (blue) consistently skews higher than the [1-10] scale (red).}
    \label{fig:supp_histograms}
\end{figure*}

\subsection{Oracle Analysis Heatmaps}
To understand the theoretical upper bound of performance, we analyze the "Oracle" performance—assuming we could perfectly select the best prediction from our ensemble of prompts and degradations.

\begin{figure}[h]
    \centering
    % Placeholder for Oracle Heatmap
    \includegraphics[width=0.8\linewidth]{example-image-c}
    \caption{Heatmap showing the frequency with which each prompt/degradation pair provides the best result in the Oracle ensemble. (Placeholder).}
    \label{fig:supp_oracle}
\end{figure}